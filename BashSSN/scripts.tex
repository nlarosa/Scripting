\documentclass{article}

\begin{document}

\title{Homework 12 - Scripts}
\author{Nicholas LaRosa}

\maketitle

\large{\textbf{Bash:}}

Bash is a pure scripting language that communicates directly with the Bourne Again Shell. 
Thus, users can call shell commands from a Bash script and are able to set environmental variables.
Additionally, functions can be created and user arguments are easily taken care of. However,
as a scripting language, the script is interpreted by the kernel, as opposed to the machine,
making execution time slow. Additionally, I find the conditional syntax in Bash to be inconsistent
and annoying.

\large{\textbf{Csh:}}

Csh, like Bash, is a scripting language that communicates directly with the C-shell. Thus,
users can call shell commands from a Csh script and are able to set environmental variables.
Csh does a nice job of allowing shell command output to be stored and searched with ease. 
Additionally, command-line arguments are handled with ease. However, arithmetic and maintaining
floating points is difficult in Csh (even moreso than Bash), and user input is strangely accepted.
Of course, as a shell script, Csh is also very slow.

\large{\textbf{Perl:}}

Perl is a higher-level scripting language than Bash or Csh, and thus can not directly call
shell commands without the use of its "system" function. Perl excels at storing data in very organized
ways (especially with hashes), and its handling of variables makes printing and passing quite easy.
Additionally, Perl can be object-oriented (but lacks data hiding), increasing its organization and
ability. However, Perl can sometimes be unreadable if too many default variables are used.

\large{\textbf{Python:}}

Similar to Perl, Python is a higher-level scripting language than Bash or Csh, and thus can not 
directly call shell commands without the use of its "os.system" function. Python also excels at 
storing data with its dictionaries and lists, with its lists being very dynamic with different types.
Python is object-oriented like Perl, but does provide more data-hiding abilities. Being a block-
oriented langauge (with statements and scope being determined by whitespace), Python achieves
better readability. However, Python variables are often somewhat unorganized, lacking any indication of type.

\end{document}
